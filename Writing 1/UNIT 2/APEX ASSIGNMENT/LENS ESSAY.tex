\documentclass{article}
\title{UNIT 2 APEX ASSIGNMENT - LENS ESSAY}
\author{Du Jinrui}
\date{\today}
\begin{document}
	\maketitle
We are not entitled. To get rights, instead of relying on other people's kindness, we must fight for them.

Racial discrimination has troubled generations of people around the world. The treatment towards African American was specially cruel and inhuman in the history of the United States. Why does it exist? Perhaps thousands of scholars has already theorized and researched it. In this essay, I analyze Martin Luther King's ``Letter from the Birmingham City Jail", form a social contract theorist' perspective, namely Hobbes' Leviathan, to argue and provide an interpretation for the cause of such discrimination in the States, that is, the creation of a social contract requires a “covenant made by every man with every man”, yet African American people were deemed outsiders, excluded from the creation of the U.S. Constitution, hence no such covenant has been made, and we are still in constant war with each other.

In order to do this, We will first extract perspectives from Leviathan to build a theoretical foundation that could help us analyzing Dr. King’s Letter. We will then inspect the Letter, and see how our theory applies to the text. Before conclusion, we will list potential counterarguments to the thesis, and refute them. Now let us examine Hobbes' ideas.

Thomas Hobbes was a political philosopher who lived the English Civil War. In the time of the war, the society was in a state of chaos: revolutions, power struggles, endless battles\ldots Hobbes' book Leviathan is an attempt to restore order, to put an end to this chaos. In the book, he talks about humans' natural tendency toward war and the reason behind it. He further explains how both rational decision made by men and their passion contribute to stopping the so called State of Nature. And he talks about Leviathan, a social contract made by people as a consequence of them trying to get out of the State of Nature. We start with the reason why people are doomed to have war against each other.

Hobbes gives three reasons. We will focus on the first two. The first reason is competing for limited resources. For example, suppose we were hunter gatherers, and in a winter, there is only one living animal we could eat. Cain successfully killed it and brought is to the cave to his family, so Abel can't. So in a night, Abel creeps into Cain's cave and killed his family so he can bring the animal to his family, or he could chain them up, enslaving them. The second reason is to defend ourselves: if Cain was awake or if he knew Abel was going to kill or enslave him, he would not let it be.

Hobbes came up with three laws for why we can't have rights for everything, or the freedom to do whatever we want for competing resources. The first one is: a world where everyone owns rights to everything is too dangerous, and there is no security for nobody. That's why we want peace. The second and the third are: we all agree to concede parts of our power to arrive a pact that everyone must comply. If anyone break this covenant, such action is unjust.

In making such covenant, the important issue is that, everyone in the society agrees upon the conceding of rights. Everyone.

Now let us move to the letter. Martin Luther King's ``Letter from the Birmingham Jail" is a response to criticisms made by a group of clergymen. It is a justification for an unlawful non-violent gathering that campaign for human rights.

He talks about the inhuman treatment that Black people received that I can't put it in more touching words: ``when you have to concoct an answer for a five year old son who is asking: `Daddy, why do white people treat colored people so mean?'; when you take a cross county drive and find it necessary to sleep night after night in the uncomfortable corners of your automobile because no motel will accept you\ldots"

He emphasizes that the demonstration will not precipitate violence, instead, it bring to light “the hidden tension that is already alive”. Moreover, “present tension\ldots is a necessary phase of transition from an obnoxious negative peace\ldots to a substantive and positive peace.”

He discusses when is a law unjust: a law is unjust when only a small number of people has devised it, for example, a number of Black people in Alabama were denied the right to vote resulting in a segregation law that is unjust to African Americans.

Now we use the lens to analyze the letter. First, we see how the mentioned two conditions of the State of Nature are met. According to the letter, White people and Black people are competing for limited resources. Black people started losing the battle since ``340 years" ago. Nowadays, they are treated badly. And we see that the demonstration is an attempt to defend themselves. So they are still in the State of Nature.

Secondly, we see how the law in Alabama or the State as a whole is unjust. According to the third lex naturalis, everyone in the society must comply to the covenant, yet in the State, a small group of people came up with the segregation law that allows only Black people to live in dire conditions. This is unjust. 

Thirdly, in generation of the covenant, everyone must participate in the making of the pact. However, the rights for vote were robbed for Black people in Alabama. Hence, this law is invalid.

Some people might argue that we can't eliminate discrimination because the human can be brutal to foreigners. It is a biological urge deeply rooted in our nature, and no social engineering can eliminate that. But Hobbes argues that both passion and reason are contributing factors for the human to end the State of Nature. By extension, the reason finds three lex naturalis that prevent future destructive behavior. One distinction between humans and animals is that humans are united by shared abstract concepts, like money, religion, or social contract. Perhaps, in the beginning, people are inclined to follow their instinct, but eventually, it’s the idea that triumphs. Also, at the beginning of his defense, Dr. King states that African Americans are not outsiders: “Injustice anywhere is a threat to justice everywhere... Anyone who lives inside the United States can never be considered an outsider anywhere within its bounds.” We can agree that sometimes a group of people tear the other side apart as if they were not humans. But as Dr. King mentioned, the
African Americans are not outsiders but Americans.

Hence, the conditions for the State of Nature are met, and people want peace, but we haven't made a successful covenant yet.




\end{document}

