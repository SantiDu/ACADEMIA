\documentclass{article}

\title{FINAL OUTLINE FOR LENS ESSAY}
\author{Du Jinrui}
\date{\today}
\usepackage[left=0.8in]{geometry}
\usepackage{paracol}
\usepackage{amssymb}
\renewcommand{\labelitemi}{$\circ$}
\renewcommand{\labelitemii}{$\centerdot$}
\begin{document}
\maketitle

\columnratio{0.2, 0.1, 0.7}
\begin{paracol}{3}
	\switchcolumn[0]
	\noindent \textbf{INTRODUCTION}
	\switchcolumn[2]
	\begin{itemize}
	\item Hook:\\
		We are not entitled. To get rights, instead relying on other people's kindness, we must fight.
	\item Motivation \& Focus:\\
		Why the establishment of Leviathan in the United States was a failing one?
	\item Thesis:\\
		I will argue that the creation of Leviathan requires a ``covenant made by every man with every man", yet African American people were deemed outsiders, excluded from the creation of the U.S. Constitution, hence no such covenant has been made, and we are still in constant war with each other.
	\item Methodology:\\
		I will compare and contrast Martin Luther King's ``Letter from the Birmingham City Jail" and excerpts from Hobbes' ``Leviathan". We will first extract perspectives from Leviathan to build a theoretical foundation that could help us analyzing Dr. King's Letter. We will then inspect the Letter, and see how our theory applies to the text. Before conclusion, we will list potential counterarguments to the thesis, and refute them.
	\item Transition
		Now let us examine Hobbes' ideas.
	\end{itemize}
\end{paracol}
\noindent\rule{16cm}{0.4pt}
%\hrulefill
\columnratio{0.2, 0.1, 0.7}
\begin{paracol}{3}
	\switchcolumn[0]
	\noindent \textbf{BODY}\\
	\noindent \textbf{PARAGRAPHS}\\
	\noindent \textbf{ON}\\
	\noindent \textbf{LENS}
	\switchcolumn[2]
\begin{itemize}
	\item Contextual information on Lens:\\
		The three Laws of Nature
		\begin{itemize}
			\item We want peace.
			\item We agree upon one covenant. The covenant says we don't have rights to everything.
			\item There should be an external power to make sure we all comply the covenant.
		\end{itemize}
		In the State of Nature, everyone is at war with everyone.\\
			The establishment of the Commonwealth depend on everyone concedes there rights to the Leviathan.
	\item Transition framing Lens + Subject:\\
		What practical issues can we examine through our lens? Let us read Martin Luther King's Letter.
\end{itemize}
\end{paracol}
\noindent\rule{16cm}{0.4pt}

\columnratio{0.2, 0.1, 0.7}
\begin{paracol}{3}
	\switchcolumn[0]
	\noindent \textbf{BODY}\\
	\noindent \textbf{PARAGRAPHS}\\
	\noindent \textbf{ON}\\
	\noindent \textbf{SUBJECT}
	\switchcolumn[2]
\begin{itemize}
	\item Contextual information on Subject
		Background information on why Dr.King wrote the letter.\\
		What is unjust law according to Dr.King.
		\begin{itemize}
			\item A law is unjust when it is not in accordance with God's law.
			\item A law is unjust when it only apply to minority.
			\item A law is unjust when it is not devised by everyone.
		\end{itemize}
	\item Transition framing how Lens will apply to Subject:\\
		How can Hobbes' theory of social contract apply to practical issue?
\end{itemize}
\end{paracol}
\noindent\rule{16cm}{0.4pt}

\columnratio{0.2, 0.1, 0.7}
\begin{paracol}{3}
	\switchcolumn[0]
	\noindent \textbf{BODY}\\
	\noindent \textbf{PARAGRAPH(S)}\\
	\noindent \textbf{ON}\\
	\noindent \textbf{LENS +SUBJECT}
	\switchcolumn[2]
\begin{itemize}
	\item Applying the Lens to the Subject
	\begin{itemize}
		\item  Points of Evidence
	\end{itemize}
\end{itemize}
\end{paracol}
\noindent\rule{16cm}{0.4pt}

\columnratio{0.2, 0.1, 0.7}
\begin{paracol}{3}
	\switchcolumn[0]
	\noindent \textbf{COUNTERARGUMENTS}
	\switchcolumn[2]
\begin{itemize}
	\item Potential counterargument \#1:\\
		The human can be brutal to foreigners. It is a biological urge deeply rooted in our nature, and no social engineering can eliminate that.
		\begin{itemize}
			\item Hobbes argues that both passion and reaso
n are contributing factors for the human to end the State of Nature. By extension, the reason finds three      lex naturalis  that prevent future destructive beha   vior.

One distinction between humans and animals is that humans are united by shared abstract concepts, like money, religion, or social contract. Perhaps, in the beginning, people are inclined to follow their instinct, but eventually, it’s the idea that triumphs.
\item At the beginning of his defense, Dr. King
 states that African Americans are not outsiders: “Inj
ustice anywhere is a threat to justice everywhere... Anyone who lives inside the United States can never be considered an outsider anywhere within its bounds."

        Yes, sometimes a group of people tear the other side apart as if they were not humans. But as Dr. King mentioned, the African Americans are not outsiders but Americans.
		\end{itemize}
	\item Potential counterargument \#2:\\
		Even though we managed to reach an agreement, such a covenant is doomed to be a bad one, for no single arrangement can satisfy the needs of everyone, and often, policy makers only make polices advantageous to powerful people.
		\begin{itemize}
			\item “… when all is reckoned
together the difference between man and man is not so considerable…" (Chapter 13, Hobbes)

                No man is powerful enough to subdue everyone, this diffidence is the source of our confidence that we can come up with a good agreement.
	\item “These rules of property and of good, ev
il, lawful, and unlawful in the actions of subjects are the civil laws…" (Chapter 18, Hobbes)

        The goal of the covenant is to restore order, not to be completely impartial.
\item “We know through painful experience that
 freedom is never voluntarily
given by the oppressor; it must be demanded by the oppressed." (King)

If we don’t fight for it, we get nothing. 
		\end{itemize}
	\item Potential counterargument \#3:\\
		There are already enough constitutional rights for all African Americans. We are at war because Black people are violent in nature.
		\begin{itemize}
			\item “All segregation statutes are
unjust because segregation distorts the soul and damages the personality… A law is unjust if it is inflic
ted on a minority that, as a result of being denied the right to vote, had no part in enacting or devising the law." (King)

All segregation orders are unjust. A law made with robbery of franchise is unjust. Thus the established laws are not good enough.
\item  ``So that in the nature of man, we find three principal causes of quarrel. First, competition; secondly, diffidence; thirdly, glory\ldots The passions that incline men to peace are: fear of death; desire of such things as are necessary to commodious living; and a hope by their industry to obtain them." (Hobbes)

        People are at war not because they love violence, but for competing for limited resources, self-defense, and glory.
\item Dr.King believes in Gandhi’s idea of nonviolence.
 He argues that the demonstration will not precipitate violence, instead, it bring to light “the hidden tensi
on that is already alive". Moreover, “present tension…
 is a necessary phase of transition from an obnoxi
ous negative peace… to a substantive and positive
peace."

The undergoing turmoil is due to the inhuman treatment, not our nature.
\item Dr.King says that there were two forces, those wh
o “do nothingism" and those who were despair, “close
to advocating violence".

        None of them implies a natural tendency towards violence.
		\end{itemize}
\end{itemize}
\end{paracol}
\noindent\rule{16cm}{0.4pt}

\columnratio{0.2, 0.1, 0.7}
\begin{paracol}{3}
	\switchcolumn[0]
	\noindent \textbf{CONCLUSION}
	\switchcolumn[2]
\begin{itemize}
	\item Reworded thesis
	\item Summary of argument
	\item Significance of the argument to your readers
	\item Open questions that still remain
\end{itemize}
\end{paracol}

\end{document}

