\documentclass{article}

\title{OUTLINE YOUR RESEARCH PAPER AND INCORPORATE A CREATIVE ELEMENT}
\author{Du Jinrui}
\date{\today}
\usepackage[left=0.8in]{geometry}
\usepackage{paracol}
\usepackage{amssymb}
\renewcommand{\labelitemi}{$\circ$}
\renewcommand{\labelitemii}{$\centerdot$}
\begin{document}
\maketitle

\columnratio{0.2, 0.1, 0.7}
\begin{paracol}{3}
	\switchcolumn[0]
	\noindent \textbf{INTRODUCTION}
	\switchcolumn[2]
	\begin{itemize}
	\item Hook $\to$* \textit{Possible creative element}:\\
	Imagine that the law enforcement obtained an device that could discover all ongoing crimes, even if it is a perfect one. Upon the detection, it punishes the offender at varying severity according to the degree of the crime. What happens if our government had this device? Will they abuse its power by making the lawful unlawful, limiting our freedom? 
	\item Motivation \& Focus:\\
	If the device is programmable, and it asks you to provide the definition of a crime, how would you define it?
	\item Thesis:\\
	I will argue that a crime is crime because it violates the freedom of people.
	\item Methodology:\\
	Because criminal activities subsume a vast range of behaviors, it's convenient to confine our discussion to certain types of them. We will draw examples from cybercrimes and cyber-terrorism, and analyze them using philosophical ideas of freedom as framework.
	\item Transition $\to$* \textit{Possible creative element}:\\
	Now we start presenting the bread of the bread and butter of this essay, by listing the cybercrimes.
	\end{itemize}
\end{paracol}
\noindent\rule{16cm}{0.4pt}
%\hrulefill
\columnratio{0.2, 0.1, 0.7}
\begin{paracol}{3}
	\switchcolumn[0]
	\noindent \textbf{BODY}\\
	\noindent \textbf{PARAGRAPHS}\\
	\noindent \textbf{ON}\\
	\noindent \textbf{CONTEXT}
	\switchcolumn[2]
\begin{itemize}
	\item Contextual information:\\
		Not all deviants are crimes.
	\item Points of Evidence:\\
		We divide the misuse and abuse of technology in to three categories: cyberdeviants, cybercrime, and cyber-terrorism.
		
		Some misuses of technology are not crimes, for example, skipping a Zoom meeting by making a fake video of yourself.

		Cybercrimes include online fraud, child pornography, cyberbullying, and more.

		The target of cyber-terrorism ranges from power grids to financial services.
\end{itemize}
\end{paracol}
\noindent\rule{16cm}{0.4pt}

\columnratio{0.2, 0.1, 0.7}
\begin{paracol}{3}
	\switchcolumn[0]
	\noindent \textbf{TRANSITION}\\
	\noindent \textbf{TO}\\
	\noindent \textbf{ARGUMENT}\\
	\switchcolumn[2]
\begin{itemize}
	\item Body paragraphs on Argument:\\
		Only those behaviors that harm the freedom of people are crimes.
	\item Contextual information:\\
		Deontological view of crime equates the crime as breach of freedom.
	\item Points of Evidence:\\
		We are not saying that a crime is crime because of its consequence that deprive the freedom of others. But crime immanently breach the freedom of others.


\end{itemize}
\end{paracol}
\noindent\rule{16cm}{0.4pt}

\columnratio{0.2, 0.1, 0.7}
\begin{paracol}{3}
	\switchcolumn[0]
	\noindent \textbf{TRANSITION}\\
	\noindent \textbf{TO}\\
	\noindent \textbf{COUNTERARGUMENTS}\\
	\switchcolumn[2]
\begin{itemize}
	\item Body paragraph(s) on Counterarguments:
	\begin{itemize}
		\item Potential counterargument \#1 $\to$* \textit{Possible creative element}:\\
			The TV series Mr. Robot tells the story of a cybersecurity engineer obliterates the financial record of everyone in the country, thus freeing them from debt. This behavior, in fact, endow people freedom instead of take it away. Why then, is it a crime?
		\item Evidence against the counterargument:\\
		The destructive behavior instigate riots, and the whole financial system must be reestablished.
		\item Explanation of why this evidence negates the counterargument:\\
			The activity falsely put the engineer himself higher than other people in the society, thus makes this behavior a crime.
	\end{itemize}
\end{itemize}
\end{paracol}
\noindent\rule{16cm}{0.4pt}

\columnratio{0.2, 0.1, 0.7}
\begin{paracol}{3}
	\switchcolumn[0]
	\noindent \textbf{POTENTIAL}
	\noindent \textbf{COUNTERARGUMENT}
	\noindent \textbf{\#2}
	\switchcolumn[2]
\begin{itemize}
	\item $\to$* \textit{Possible creative element}:\\
	\item Evidence against the counterargument:\\
	\item Potential counterargument \#3:\\
	\item Explanation of why this evidence negates the counterargument:
\end{itemize}
\end{paracol}
\noindent\rule{16cm}{0.4pt}

\columnratio{0.2, 0.1, 0.7}
\begin{paracol}{3}
	\switchcolumn[0]
	\noindent \textbf{CONCLUSION}
	\switchcolumn[2]
\begin{itemize}
	\item Reworded thesis:\\
	A crime in essence breaches the freedom of all the people in society.
	\item Summary of argument:\\
		While some cyberdeviants are not crimes, cybercrime and cyber-terrorism harm our freedom. Hence, they are crimes.
	\item Significance of the argument to your readers $\to$* \textit{Possible creative element}:\\
	Most people in this modern society are prone to get harmed by cybercrimes. It is beneficial to identify these crimes, so we could report in time to help reduce the crimes.
	\item Open questions that still remain:\\
	What causes these crimes?
\end{itemize}
\end{paracol}

\end{document}

